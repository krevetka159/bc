\chapter{Programátorská dokumentace}


\section{Objektový návrh}

Obrázek jak spolu interagují jednotlivé třídy (diagram).

\section{Hlavní komponenty}
\subsection{Game}
Samotná hra je reprezentovaná třídou Game, která obsahuje instanci hráče (lidského nebo počítačového), instanci třídy Bag, která implementuje množinu všech herních dílků ve hře, a nabídku dílků k přiložení. Třída Game obstarává celý průběh hry. Pro zobrazení stavu hry využívá instanci třídy GameStatePrinter. 

Ze třídy Game dědí třída SimulationGame, která reprezentuje běh simulaci hry ve stromovém prohledávání od určitého stavu do konce.

\subsection{Player}
Třída Player reprezentuje hráče. Obsahuje instanci třídy GameBoard a metody, které obstarávají výběr dalšího tahu.
Z této třídy dědí třída Agent, která navíc implementuje metody na náhodný výběr dílku a pozice pro další tah.
Z třídy Agent dále dědí všechny třídy implementující agenty.

\subsection{GameBoard}
Třída GameBoard spravuje herní pole. Herní deska je inicializovaná s okraji, které se rovněž počítají při spojování dílků do clusterů se stejnými vlastnostmi.
Obsahuje metody pro kontrolu sousedních polí a vyhodnocení užitku přidání herního dílku na určitou pozici na herní desce, které jsou používané agenty při vyhodnocování akcí v rámci jejich strategie. Třída obsahuje instanci třídy ScoreCounter, která je zodpovědná za průběžné počítání skóre ve hře.

\subsection{Evolution}
Třída Evolution obstarává celý běh evolučního algoritmu. Obsahuje metody pro základní části evolučního algoritmu, tedy inicializaci generace, mutaci, selekci a výpočet hodnoty fitness. Jedinec je instancí třídy Weights. 

\section{Počítání skóre}
\subsection{Využití UnionFind}

Union Find pro barvy a vzory, úkolové dílky jsou vyhodnocovány samotnými úkolovými dílky, které si uchovávají seznam dílků, které s nimi sousedí.
\textbf{ozdrojovat průvodce}

\section{Implementace strategií agentů}