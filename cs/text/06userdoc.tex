\chapter{Uživatelská dokumentace}

\todo[inline]{Program je spustitelný .exe souborem or sth.
Konzolová aplikace je určena zejména pro testování různých nastavení hry a strategií počítačového hráče. Uživatel si také v aplikaci může vyzkoušet hru v sólové verzi. Na obrázku \ref{fig:konzole} vidíme počáteční stav konzole po spuštění a základní módy, ze kterých si uživatel může vybrat. Výběr módu uživatel provede zadáním čísla, které odpovídá danému módu.}

\figureKonzole

Aplikace provází uživatele výběrem módu a zadáním potřebných i volitelnými parametry pomocí vysvětlivek. Na obrázku xy můžeme vidět, jakým způsobem uživatel s konzolí komunikuje? idk.

\section{Testovací módy}

V rámci aplikace máme k dispozici několik testovacích módů. 

Pro všechny testovací módy si má uživatel možnost zvolit, zda chce výsledky testu vytisknout do souboru. Pokud ano, program ho požádá o vyplnění názvu souboru. Pokud ne, program výsledek pouze vytiskne do konzole.

\subsection{Testování agentů}
Pro testování agentů zvolíme, kterého z agentů chceme testovat. V konzoli uvidíme seznam agentů, které jsme popsali v předchozí kapitole. Pro výběr agenta zadáme číslo, které mu odpovídá. Na základě výběru agenta program uživatele požádá o vyplnění dalších parametrů.

Pro jednokrokové agenty není potřeba zadat žádné parametry testování. V případě testování agentů, kteří využívají stromové prohledávání bez simulací je potřeba zadat limit pro hloubku stromu a discount faktor. V případě prohledávání se simulacemi je potřeba zadat počet her, které budou tvořit jednu simulaci. 

V případě, že uživatel zvolí možnost vypsání výsledků do souboru, se do souboru data zapíší ve formátu běžném pro .csv soubor. Na obrázku xy vidíme soubor, ve kterém jsou zapsána data z testování náhodného agenta. Jedna řádka souboru odpovídá jedné hře, kterou agent odehrál. Pro každou hru program uchová informace o dosaženém skóre, získaných žetonech a splněných úkolech.

\textbf{Zapsat co znamenají ty jednotlivé sloupce}


\subsection{Evoluční algoritmus}

Po zvolení módu pro evoluční algoritmus uživatel zadá, pro které úkolové dílky chce optimalizovat. Dále zvolí, zda chce tuto kombinaci použít s konkrétním umístěním. Při výběru této možnosti umístění odpovídá pořadí, v jakém byly úkoly uživatelem zapsány. První úkol bude umístěn na místo ve třetím řádku, druhý úkol na místo ve čtvrtém řádku a poslední úkol na místo na pátém řádku herní desky.

Výsledek evolučního algoritmu je potřeba vytisknout do souboru. Do uživatelem zvoleného souboru program po ukončení optimalizace vah do souboru vypíše všechny jedince finální generace s hodnotou jejich fitness funkce jak je zobrazeno na obrázku 420.

\subsection{Testování nastavení hry}

V rámci testování různých nastavení hry má uživatel možnost otestovat výsledky agenta s neváženou rozšířenou ohodnocující funkcí (agent č. 5) pro pevně dané úkolové dílky a číslo herní desky. Pro úkolové dílky taky může zvolit, zda je chce testovat s pevně daným umístěním na desce. Oba tyto parametry jsou volitelné, v případě, že uživatel nechce zadat konkrétní nastavení, program otestuje všechny dvojice herní desky a kombinace úkolů.

Výsledkem testování pro dané nastavení je pouze průměrné skóre, kterého agent s daným nastavením dosáhl. Výpis dat do souboru je volitelný.

\section{Herní mód}

Barvy jsou kódovány čísly 1-6, vzory velkými písmeny A-F.
Prázdná políčka jsou označena -{}-.
Políčka označená XX jsou určená pro úkolové dílky.

Na začátku hry uživatel postupně vybere, které úkolové dílky chce umístit na svou hrací plochu. 

\figureUkoly

V každém kole hry se v konzoli znovu vytiskne stav hry. Nějaký obrázek kde je popsané co je co
1. Dílky k použití
2. Bodování pro jednotlivé vzory
3. Průběžné skóre
4. Hrací deska
5. Instrukce pro provedení dalšího tahu

Pro přidání herního dílku na hrací desku hráč zadá číslo 1-3 které odpovídá dílku v nabídce a následně zadá číslo řádku a sloupce pozice, na kterou chce daný dílek umístit.

\figureSinglePlay