%%% Fiktivní kapitola s ukázkami sazby

\chapter{Popis hry}

\section{Základní pravidla}

Hra Calico je založena na principu postupného přikládání dílků. 

Každý hráč má vlastní herní desku s políčky v mřížce velikosti 5x5. Na desku postupně doplňuje dílky s různými barvami a vzory a získává body podle zadaných pravidel na seskupování barev a vzorů v mřížce. Hráči se při vybírání dílků střídají. Hra končí, když mají všichni hráči vyplněné celé herní pole.

Vítězí hráč, kterému se podaří získat nejvíce bodů.

\subsubsection*{Herní materiál???}
6 barev, 6 vzorů

Hra obsahuje 3 dílky od každé kombinace barvy a vzoru (36 kombinací), tedy 108 herních dílků.
\subsubsection*{Začátek hry}
Každý hráč dostane svou hrací desku.

Na stůl se vyloží 3 dílky z pytlíku. 

Náhodně se vybere, ze kterých vzorů budeme skládat trojice/čtveřice/pětice.
\subsubsection*{Průběh hry}

V každém kole hry se všichni hráči vystřídají na tahu ve stejném na začátku stanoveném pořadí. Hráč na tahu vybere 1 ze 3 možných dílků a přidá ho na svou hrací desku na libovolné neobsazené políčko. Poté doplní možnosti použitelných dílků.

\subsubsection*{Podmínky získání bodů}
Za každou trojici sousedících stejnobarevných dílků hráč získává 3 body.

Vzory jsou náhodně rozděleny do 3 dvojic s následujícím bodováním:
\begin{enumerate}
    \item Sousedící trojice se stejným vzorem = 3 body
    \item Sousedící čtveřice se stejným vzorem = 5 bodů
    \item Sousedící pětice se stejným vzorem = 7 bodů
\end{enumerate}

\subsubsection*{Konec hry}

Hra končí když každý hráč naplní celé své hrací pole.


\section{Možná rozšíření}

Tohle je nějaká family verze -> víc casual. Běžně se hraje s úkolovými dílky.

př. těch úkolových dílků

možnost specifikovat tvar skupin dílků se stejným vzorem

víc her za sebou, každá s jiným pravidlem (např. bez zisku knoflíků)

\section{Mé úpravy oproti původním pravidlům}

bez žetonků kočiček a knoflíků, takže nekontroluji víc kočiček vedle sebe, nebo jestli je nový barevný cluster skutečně kompaktní (např 6 dílků = 2 clustery bez ohledu na to, jestli ta druhá trojice sousedí), protože by to bylo náročné na rozhodování v případě sjednocení více clusterů, který z nich chci vzít

automatické počítání bodů v průběhu hry 

body za posbírání knoflíků všech barev ?? (můžu přidat)

