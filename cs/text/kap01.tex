%%% Fiktivní kapitola s ukázkami sazby

\chapter{Popis hry}

Hra Calico je založena na principu postupného přikládání dílků. 

Každý hráč má vlastní herní desku s políčky v mřížce velikosti 5x5 z nichž 3 políčka jsou určena pro speciální dílky rozšíření hry. Na desku postupně doplňuje dílky s různými barvami a vzory a získává body podle zadaných pravidel na seskupování barev a vzorů v mřížce. Hráči se při vybírání dílků střídají. Hra končí, když mají všichni hráči vyplněné celé herní pole.

Vítězí hráč, kterému se podaří získat nejvíce bodů.
\todo[inline,color=red!40]{Možná trochu upravit - pročíst!}

\section{Pojmy}

\todo[inline]{Nějaká věta že pojmy}

\begin{itemize}
    \item Útržky - šestiúhelníkové dílky, které hráči pokládají na své herní desky. Každý útržek má svou barvu a vzor. Hra obsahuje 3 útržky pro každou kombinaci 6 barev a 6 vzorů, tedy 108 útržků.
    \item Úkolový dílek - Úkolové dílky se přikládají na 3 speciálně určená místa na hrací desce. Specifikují podmínky pro přikládání útržků na pozice s úkolovým dílkem sousedící hranou. Hráč tyto podmínky nemusí splnit, ovšem za~jejich splnění získává body.
    \item Dílky vzorů - Dílky vzorů jsou černobílé útržky, pro každý vzor existuje právě jeden. Slouží k určení bodování pro jednotlivé vzory. Přikládají se k panelům koček.
    \item Panel kočky - Panel kočky je karta, která slouží k určení bodování pro jednotlivé vzory. Každý panel určuje podmínku sdružování vzorů a odpovídající bodové ohodnocení. Podmínka je buď minimální počet sousedících útržků stejného vzoru, nebo přímo tvar, který mají útržky tvořit. K panelům koček se přikládají dílky vzorů.
    \item Pelíšek - Jako pelíšek je označena souvislá plocha útržků se stejným vzorem splňující minimální počet útržků podle odpovídajícího panelu kočky.
    \item Žeton kočky - Žetony s obrázky koček sloužící k~označování částí hrací desky, za které hráči získávají body. Přikládají se na již přiložené útržky.
    \item Přilákání kočičky - Hráč kočičku přiláká tím, že splní podmínky panelu kočky. Poté si na hrací desku přidá žeton kočky, který danému panelu odpovídá.
    \item Knoflík - Knoflíky jsou barevné žetony v barvách útržků, které slouží k~označování částí hracího pole, za které hráči získávají body. Přikládají se na již přiložené útržky.
    \item Duhový knoflík - Duhový knoflík je žeton, který hráč získá, pokud získá knoflík pro každou ze 6 barev.
\end{itemize}

\section{Základní pravidla}

\subsubsection*{Začátek hry}

Každý hráč dostane svou hrací desku a stejnou sadu 6 úkolových dílků. Na políčka pro úkolové dílky hráč položí 3 úkolové dílky dle svého výběru. 
Na stůl se položí 3 panely koček - pro seskupení 3 a více, 4 a více a 5 a více stejně vzorovaných dílků bez přesně určeného tvaru seskupení. Každému panelu kočky náhodně přidělíme 2 černobílé dílky vzorů. 
Bank knoflíků a kočiček podle použitých panelů necháme v dosahu všech hráčů. 
Všechny útržky se vloží do sáčku a zamíchají. Každý hráč si vezme 2 útržky ze sáčku a vezme si je do ruky, ostatním je neukazuje. Na stůl se vyloží nabídka 3 útržků.

\subsubsection*{Průběh hry}

V každém kole hry se všichni hráči vystřídají na tahu ve stejném na začátku stanoveném pořadí. Hráč na tahu vybere 1 z dílků, které má v ruce, a přidá ho na svou hrací desku na libovolné neobsazené políčko. Pokud má nárok na knoflík nebo kočičku, položí dané žetony na svou hrací plochu podle pravidel přikládání žetonů. Poté si vezme do ruky dílek z banku a bank doplní útržkem z pytlíku.

%\subsubsection*{Pravidla pro získání a přikládání žetonů}
\subsubsection*{Bodování knoflíků}
Za každý knoflík hráč získá 3 body. Knoflíků stejné barvy může hráč získat víc, nově vzniklá plocha alespoň 3 útržků však musí být oddělená od ploch, za které již hráč knoflík této barvy získal. Pokud hráč během hry spojí více ploch, na které již knoflíky umístil, dané knoflíky mu zůstávají.

\subsubsection*{Bodování koček}
Panely koček jsou ohodnoceny 3, 5 a 7 body.
Pro přilákání kočiček platí podobná pravidla jako pro knoflíky, pro přilákání kočičky vzorem, pro který jsme již vytvořili pelíšek, je potřeba vytvořit plochu, která s daným pelíškem nesousedí. Dva pelíšky různých vzorů ze stejného panelu kočky však sousedit mohou. \todo[inline,color=red!40]{přepsat}

\subsubsection*{Bodování úkolových dílků}
Úkolové dílky mají dvě možnosti bodového ohodnocení - za částečné a úplné splnění. Částečné splnění znamená splnění podmínky pouze vzorem nebo pouze barvou, úplné splnění pak znamená splnění podmínky barvou i vzorem. Každý úkolový dílek má jiné bodové ohodnocení, které je specifikováno přímo na dílku.


\subsubsection*{Konec hry}

Hra končí, když všichni hráči naplní celé své hrací desky, tedy po 22 kolech. Vyhrává hráč s~nejvyšším součtem bodů. Hra nabízí více verzí, které je také možné kombinovat s dalšími možnostmi rozšíření.


\section{Varianty hry a možná rozšíření}

Pravidla popsaná v předchozí sekci platí pro základní hru ve zjednodušeném režimu.

\subsubsection*{Základní pravidla v plné verzi}
Tato verze nabízí hráčům volbu panelů koček. Hráči tedy můžou volit mezi více možnostmi podmínek pro shlukování vzorů. Panely se složitějšími podmínkami jsou ohodnoceny více body.

\subsubsection*{Sólová hra}
Všechny varianty lze hrát ve verzi pro jednoho hráče, avšak absence protihráčů způsobuje nepříliš velkou obměnu možností útržků. Sólová varianta hry tento aspekt řeší tím, že po každém svém tahu hráč zahodí ten útržek z banku možností, který je více vpravo. Tuto verzi implementuje webová aplikace. \todo[inline,color=red]{Odkaz na webovou aplikaci}

\subsubsection*{Další možné úpravy}
Hráči mohou dalšími možnostmi úprav měnit náročnost hry. Pro jednodušší hru je možné využít tzv. rodinnou verzi, ve které hráči neplní úkoly. Na začátku hry si na pole herní desky, které jsou určená pro úkolové dílky, položí úkolové dílky lícem dolů. Na tato místa se tedy nedají přikládat hrací dílky a počet kol zůstává oproti běžné verzi stejný. Pro snížení počtu dílků ve hře je možné odstranit jednu barevnou sadu dílků (36 dílků). Hráči se však také mohou domluvit na zpřísňujících pravidlech pro hru jako jsou např. různá omezení pro počet získaných knoflíků a žetonů koček.

Dále také mohou hráči upravovat herní mechaniky. Jako příklad lze uvést hru, ve které hráči nemají žádné útržky v ruce, ale ve svém tahu přikládají na svou herní desku přímo dílek z trojice dílků uprostřed stolu. Každý tah hráče pak může být omezen pravidlem, že pokud je to možné, musí přiložit takový dílek, který lze přiložit k dílku stejné barvy nebo stejného vzoru.

\begin{itemize}
    \item plnění milníků při více hrách za sebou, např. hráč nesmí za 1 hru získat žádný knoflík, hráč musí získat po 2 žetonech každé kočky ve hře, hráč musí získat více než určený počet bodů
\end{itemize}

\section{Mé úpravy oproti původním pravidlům}

používám základní hru ve zjednodušeném režimu spolu s rozšířením, ve kterém hráči nemají dílky v ruce, ale přikládají rovnou dílek ze tří možností na stole

automatické počítání bodů v průběhu hry -> neřeším pravidla pokládání žetonků na herní desku

