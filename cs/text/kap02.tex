\chapter{Analýza hry}

\section{Vlastnosti hry?}
vidím všechno - herní desky ostatních, stejné možnosti pro každého

hra s úplnou informací (alespoň v mojí verzi)

hra s nenulovým součtem

sekvenční hra (hráči se střídají na tahu)

intenzivní množství pravděpodobnosti, což je bída

mám přesně daný počet kol = 22, konečná hra


\section{AI - přístupy ?}
jaké přístupy pro ai dávají a nedávají smysl

basic tree search, monte carlo tree search

evoluce
\subsection{Strom}

V botanice je strom vytrvalá rostlina s prodlouženým dřevnatým stonkem nebo kmenem, který u většiny druhů nese větve a listy. V některých případech může být definice stromu užší a zahrnuje pouze dřeviny s druhotným růstem, rostliny využitelné jako dříví nebo rostliny nad určitou výšku. V širších definicích jsou stromy také vyšší palmy, kapradiny, banány a bambus. Stromy nejsou taxonomickou skupinou, ale zahrnují různé druhy rostlin, které si nezávisle vyvinuly kmen a větve jako způsob, jak se tyčit nad jinými rostlinami a soutěžit o sluneční světlo. Stromy bývají dlouhověké, některé dosahují stáří několika tisíc let. Stromy na Zemi existují již 370 milionů let. Odhaduje se, že na světě jsou přibližně tři biliony vzrostlých stromů. - Wikipedia

\subsection{MonteCarlo treesearch}

Monte Carlo Tree Search (MCTS) je algoritmus často užívaný v rozhodovacích procesech, zejména v teorii her a jiných případech, kdy je potřeba prohledávat velký stavový prostor. Algoritmus je schopný provádět informovaná rozhodnutí i v komplexním prostředí s velkou mírou nejistoty.

Stromová reprezentace hry

Selekce

Expanze

Simulace

Zpětný chod

\subsection{Evoluční algoritmus}
 
Evoluční algoritmy jsou optimalizační algoritmy inspirované biologickou evolucí. Pracujeme s množinou kandidátů na řešení daného problému, těmto kandidátům říkáme jedinci a množinu jedinců označujeme jako populaci. 

Jedna iterace evolučního algoritmu se nazývá generace. V každé generaci algoritmu vyhodnotíme tzv. fitness funkci každého jedince, která určuje, nakolik je řešení problému reprezentované daným jedincem dobré. Generace se typicky skládá z následujících částí.

\subsubsection*{Selekce}

Z aktuální populace potřebujeme vybrat jedince, ze kterých následně vytvoříme jedince pro novou generaci. Takovým jedincům říkáme rodiče, nové generaci pak jejich potomci. Příkladem může být turnajová selekce, kde s pravděpodobností $p_t$ ze dvou náhodně vybraných jedinců vybereme jako rodiče toho,  který má lepší hodnotu fitness.

\subsubsection*{Aplikace genetických operátorů}

Na rodiče vybrané z aktuální generace dále aplikujeme genetické operátory. 
Operace křížení kombinuje dva jedince pro vytvoření jedince nového. Příkladem může být jednobodové křížení, kde náhodně vybereme jeden bod v jedinci a nový jedinec vznikne spojením dvou částí různých jedinců (rodičů) rozdělených daným bodem. 
Operace mutace vytvoří nového jedince tím, že náhodně upraví některé parametry již existujícího jedince. Např. otočí bit nebo prohodí bity na dvou pozicích v jedinci.

\subsubsection*{Přenos jedinců do další generace}


