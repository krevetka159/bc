\chapter{Implementace agentů}

rozdělení na single a multiplayer

možná evoluce? pak rozdělení na ručně psané a optimalizující idk

monte carlo tree search

rozdělení hry na fáze?? spíš ne

\section{Značení}
\begin{align*}
    &B \in \Bbb{Z} &\text{bodové ohodnocení za zisk knoflíku}\\
    &V \in \Bbb{Z}^3 &\text{bodové ohodnocení za zisk žetonu kočky (3 možnosti)}\\
    &V_v \in \Bbb{Z} &\text{bodové ohodnocení zisku kočky pro vzor } v\\
    &T \in \Bbb{Z}^{6\times2} &\text{??? možná jinak}\\
    &x &\text{herní dílek}\\
    &x_b &\text{barva herního dílku}\\
    &x_v &\text{vzor herního dílku}\\
    &b_{(x,(i,j))} \in \{0,1,2\} &\text{počet knoflíků získaných přidáním dílku $x$ na pozici $(i,j)$}\\
    &v_{(x,(i,j))} \in \{0,1\} &\text{$1$ pokud přidáním dílku $x$ na pozici $(i,j)$ získá žeton kočky}\\
    &N_b(i,j) \in \{0,1\} &\text{$1$ pokud je na pozici sousedící s $(i,j)$ dílek barvy b}\\
\end{align*}

\section{Ručně psaní agenti}
\subsection{Hra pro jednoho hráče}

Náhodný agent - vloží náhodně vybraný dílek z nabídky na náhodnou pozici na herní desce. Hlavně pro porovnání.

\vspace{10pt}

Nejlepší umístění dílku vzhledem k barvám.

$u_b(x,(i,j)) = N_{x_b}(i,j)*(B*b_{(x,(i,j))} + 1)$

\vspace{10pt}

Nejlepší umístění dílku vzhledem ke vzorům.

$u_v(x,(i,j)) = N_{x_v}(i,j)*(V_{x_v}*v_{(x,(i,j))} + 1)$

\vspace{10pt}

Nejlepší umístění dílku.

$utility(x,(i,j)) = u_b(x,(i,j)) + u_v(x,(i,j)) + úkoly$

\vspace{10pt}

 Přidání malé náhody - užitková fce, ale s malou pravděpodobností random agent.

\vspace{10pt}

Nejlepší umístění náhodně vybraného dílku. Pro porovnání s multiplayerem, kde se budu snažit umístit dílek, který je nejlepší pro soupeře.

\vspace{10pt}

Vylepšená utility fce, aby zohledňovala vytváření souvislých ploch, které vedou k získání žetonu / splnění tasku, ale nejsou tím posledním dílkem, kterým toho skutečně dosáhneme.


\subsection{Hra pro více hráčů}

agent, který se snaží nejlépe použít dílek nejlepší pro soupeře - souvisí s agentem pro nejlepší umístění náhodného dílku - špatné i se soupeřem

agent kombinující tu lepší utility fci a škodění soupeři

\section{Evoluční algoritmus}
\subsection{Evolvované parametry}

Používáme funkci pro ohodnocení stavu, kterou používáme v (nějakém?) agentovi, ale chceme každému aspektu ohodnocení přiřadit různé váhy. Upravená funkce pro evoluci tedy vypadá následovně:

$b*color\_utility(x) + c*pattern\_utility(x) + t*task\_utility(x)$ 

$b \in \Bbb{R}, c \in \Bbb{R}^3, t \in \Bbb{R}^3$

x = dílek, jehož přidáním se dostaneme do nového stavu

\subsection{Průběh evoluce}

\subsubsection*{Incializace}
Náhodná inicializace vah - náhodná čísla z rozmezí $(0,1)$
\subsubsection*{Fitness funkce}
Průměrné skóre z 500 her agenta, který se rozhoduje ohodnocením následujícího stavu pomocí užitkové funkce s danými vahami. Všechny hry používají stejnou herní desku se stejně rozloženými úkolovými dílky.
\subsubsection*{Selekce}
Algoritmus používá turnajovou selekci s hranicí nastavenou na 80\%, tedy ze dvou náhodně vybraných jedinců z populace s pravděpodobností 80\% vybereme do nové populace toho, který dosáhl lepších výsledků.
\subsubsection*{Křížení}
Křížení v algoritmu nepoužívám, protože prostě ne, nevím.
\subsubsection*{Mutace}
Mutaci aplikujeme na každý parametr každého jedince. Využíváme Gaussovskou mutaci, tedy přičítáme náhodné číslo z normálního rozdělení se střední hodnotou 0 a rozptylem 0,01.

\subsection{Nastavení parametrů evoluce}



